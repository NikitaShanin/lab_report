%%% Поля и разметка страницы %%%
\usepackage{lscape}		% Для включения альбомных страниц
\usepackage{geometry}	% Для последующего задания полей

%%% Кодировки и шрифты %%%
\usepackage{cmap}						% Улучшенный поиск русских слов в полученном pdf-файле
\usepackage[T2A]{fontenc}				% Поддержка русских букв
\usepackage[utf8]{inputenc}				% Кодировка utf8
\usepackage[english, russian]{babel}	% Языки: русский, английский
\usepackage{pscyr}						% Красивые русские шрифты
\usepackage{hhline}	
\usepackage{lastpage}

%%% Математические пакеты %%%
\usepackage{amsthm,amsfonts,amsmath,amssymb,amscd} % Математические дополнения от AMS

%%% Оформление абзацев %%%
\usepackage{indentfirst} % Красная строка

%%% Цвета %%%
\usepackage[usenames]{color}
\usepackage{color}
\usepackage{colortbl}

%%% Таблицы %%%
\usepackage{longtable}					% Длинные таблицы
\usepackage{multirow,makecell,array}	% Улучшенное форматирование таблиц

%%% Общее форматирование
\usepackage[singlelinecheck=off,center]{caption}	% Многострочные подписи
\usepackage{soul}									% Поддержка переносоустойчивых подчёркиваний и зачёркиваний

%%% Библиография %%%
\usepackage{cite} % Красивые ссылки на литературу

\usepackage{ifthen}                 % добавляет ifthenelse

% настройка подписей к рисункам и таблицам
\usepackage{caption}
\usepackage{subcaption}

%%% Гиперссылки %%%
\usepackage[linktocpage=true,plainpages=false,pdfpagelabels=false]{hyperref}

%%% Изображения %%%
\usepackage{graphicx} % Подключаем пакет работы с графикой

%%% Оглавление %%%
\usepackage{tocloft}

%%% Подписи %%%
\usepackage{caption}                                % Для управления подписями (рисунков и таблиц) % Может управлять номерами рисунков и таблиц с caption %Иногда может управлять заголовками в списках рисунков и таблиц
\usepackage{subcaption}                             % Работа с подрисунками и подобным

\usepackage{textcomp}
\usepackage{calc}

\usepackage{titlesec} % для titleformat
\usepackage{interfaces-base}

% для номеров страниц
\usepackage{fancyhdr} % пакет для установки колонтитулов

%%% Счётчики %%%
\usepackage[figure,table]{totalcount}               % Счётчик рисунков и таблиц
\usepackage{totcount}                               % Пакет создания счётчиков на основе последнего номера подсчитываемого элемента (может требовать дважды компилировать документ)

\usepackage{todonotes}

\RequirePackage{keyval}
\usepackage{ifthen}

\usepackage{amsmath,amssymb}
\usepackage{enumitem}
\usepackage{textpos}

